\documentclass[11pt]{article}

\usepackage{graphicx}
\usepackage{fancyref}
\usepackage{fancyhdr}

\title{Physics 426 - Instability Notes}
\author{Jody Klymak}

\begin{document}

\maketitle
\pagestyle{fancy}
\section{Introduction}

\begin{figure}[hbtp]
  \begin{center}
    \includegraphics[width=4in]{images/Instability.jpg}
    \caption{Shear instability in a fluid; the flow in the upper half is moving
to the right faster than the fluid in the lower half.   The three panels are
distance downstream.}
    \label{fig:Instability}
  \end{center}
\end{figure}

We have seen in the lab demos that fluid flow is very susceptible to becoming
unstable and breaking down into turbulence.  An example is shown in
\fref{fig:Instability} where the flow in the upper layer is moving faster than
flow in the lower layer and this leads to a "shear instability".  A few things
to note:

\begin{itemize}
  \item There is clearly a preferred horizontal scale to this instability.  The
billows that grow are one size until they self-interact and become turbulent.
  \item This horizontal scale must be present in the upstream conditions, but
it need not be large.  
  \item The instability takes time to grow. 
\end{itemize}

This leads to a few questions we commonly ask ourselves about a flow
\begin{enumerate}
  \item Is a flow unstable? 
  \item If a flow is unstable can we say what scale has the fastest growth
rate?
  \item How fast is that growth rate?   
\end{enumerate}

\section{Rayleigh Shear Instability}

As a simple situation that often leads to instability lets consider a
homogeneous steady background flow $u = U(y)$.  The idea of an instability is
that there are always small imperfections to this flow.  Physically, its
easiest to think of these as small displacements of the streamlines around the
mean and then consider what happens if there are small perturbations to the
flow.  If the flow is stable, then all perturbations will be damped, whereas if
it is unstable some perturbations will grow. 


\begin{figure}[hbtp]
  \begin{center}
    \includegraphics[width=5in]{images/ShearInstabilitySketch.png} 
    \caption{Schematic of shear instability.  Background flow is steady and has
a dependence in $y$. The idea of a flow instability is to perturb the flow and
see what scales in the flow grow.   }
    \label{fig:Instability}
  \end{center}
\end{figure}

\subsection{Linear perturbation analysis}

All of these methods decompose the flow into a background component of the flow
and a 'perturbation', and then assume that the perturbation is small so that
only linear terms in the perturbation are retained.  i.e. if $u = U(y) + u'(x,
y, y)$ then terms that are quadratic or higher in $u'$ are assumed small.   So,
similarly $v=v'$, and $p=p'$ because neither of these two variables have a
background component. 

So for the flow above the x-, and y-momentum equations, and continuity
equations become:

\begin{eqnarray*}
  \frac{\partial u'}{\partial t} + U\frac{\partial u'}{\partial x} +
v'\frac{\partial U}{\partial y} & = & - \frac{1}{\rho}\frac{\partial
p'}{\partial x} \\
  \frac{\partial v'}{\partial t} + U\frac{\partial v'}{\partial x} & = & -
\frac{1}{\rho}\frac{\partial p'}{\partial x}\\
  \frac{\partial u'}{\partial x} + \frac{\partial v'}{\partial y} & = & 0
\end{eqnarray*}

To see how this flow behaves in the presence of perturbations, we assume the
flow is a periodic travelling wave in the x-direction and has a y-direction
structure (to be determined):

\begin{eqnarray*}
  u' &= &\hat{u}(y)\ e^{i k (x -ct)}\\
  v' &= &\hat{v}(y)\ e^{i k (x -ct)}\\
  p' &= &\hat{p}(y)\ e^{i k (x -ct)}\\
\end{eqnarray*}

This gives three coupled PDEs for the vertical structure of the response that
depends on $k$ and $c$:

\begin{eqnarray*}
-ikc\ \hat{u} + ik U\ \hat{u} + U'\ \hat{v} & = & -ik\ \hat{p}\\
-ikc\ \hat{v} + ik U\ \hat{v}  & = & -ik\ \frac{\partial \hat{p}}{\partial y}\\
ik\ \hat{u} + \frac{\partial \hat{v}}{\partial y} & = & 0
\end{eqnarray*}
where here we have defined $U' \equiv \frac{\partial U}{\partial y}$.

This is three coupled equations in three unknowns, so as usual we combine,
choosing to do so for $\hat{v}(y)$, and after some algebra get

\begin{equation}
\label{eq:Rayleigh}
  \hat{v}'' + \hat{v}\left(\frac{U''}{c-U} - k^2\right) = 0
\end{equation}

How does this help us?  \Fref{eq:Rayleigh} cannot be solved analytically, but
as we will see below some useful criteria for stability can be derived
analytically using it.  However, first what is the character of the numerical
solutions?  We note that $k$ is always real (its the wavelength of the
disturbance we are interested in), but that $c$ may be imaginary in general. 
If it is imaginary, then we have growing solutions if $\mathcal{I}(c) < 0$
(assuming $k>0$) and the growth rate will be given by $k\mathcal{I}(c)$.

Now all we need to do is solve for $c$ and $\hat{v}(y)$ for a given $U(y)$ and
$k$. \Fref{eq:Rayleigh} is a classic eigenvalue problem, where $c$ can have a
number of distinct values for a given background profile of $U(y)$.  When we
solve this numerically, the numerical solution ends up being a matrix
eigenvalue problem.  

\subsection{Numerical Solution}

Numerically \fref{eq:Rayleigh} is straightforward to set up using first
differencing.  First we rewrite in an eigen-vector/eigen value form, where
$\hat{v}(y)$ are the eigen vectors and $c$ are the eigenvalues:
\begin{equation}
  \left[ \left(\frac{\partial^2}{\partial y^2} - k^2\right) U -
U''\right]\hat{v} = c \left[ \frac{\partial^2}{\partial y^2} - k^2\right]
\hat{v} 
\end{equation}
We write it this way because that allows us to write the discretized form as
matrix equation of the form 

\begin{equation}
 A \mathbf{v} = c B \mathbf{v}
\end{equation}
where $A$ and $B$ are matrices, and $\mathbf{v}$ is an eigenvector we are
solving for and $c$ is a \emph{generalized} eigenvalue problem for which there
are numerical solvers in most linear algebra packages (i.e.
\texttt{scipy.linalg.eigh}).  

Setting up the matrices $A$ and $B$ is straight forward based on our previous
discretization attempts.  If divide the y-domain into $N$ grid points, indexed
from $0$ to $N-1$, then we end up with two tri-diagonal matrices: 
\begin{eqnarray*}
  A_{i, i-1} = A_{i, i+1} & = & U_i / \Delta y^2\\
  A_{i, i} & = &  \left(-2 /\Delta y^2 -k^2\right)U_i -(U_{i+1} + U_{i-1} -2
U_i)/\Delta y ^2\\
\end{eqnarray*} 
and for $B$:
\begin{eqnarray*}
  B_{i, i-1} = B_{i, i+1} & = & 1 / \Delta y^2\\
  B_{i, i} & = &  -2 /\Delta y^2 -k^2 \\
\end{eqnarray*} 
These relationships are for interior points where $i>0$ and $i<N-1$.  For the
first and last row we need to assume boundary conditions, which in this case is
the disturbance goes to zero at the boundaries: $\hat{v}_{-1} = \hat{v}_{N} =
0$, and assume that $U$ has not gradient there: $U_{-1} = U_0$ and $U_{N} =
U_{N-1}$.  If we do that then we just plug those values into the above
relations to get 
\begin{eqnarray*}
A_{0, 0} &=& \left(-2 /\Delta y^2 -k^2\right)U_0 -(U_1 - U_0)/\Delta y ^2\\
A_{0, 1} & = &U_0/\Delta y^2\\
B_{0, 0} & = &  -2 /\Delta y^2 -k^2 \\
B_{0, 1} & = & 1 / \Delta y^2
\end{eqnarray*}
and similarly for $i=N-1$.   

This eigenvalue problem can then be solved for each value of $k$ to be
explored, and the fastest growing unstable mode determined for each $k$.  An
example of this calculation is shown for the profile in
\fref{fig:VelocityProfile}.  There is a clear minimum in the growth time for
this profile at $k=2.3\times10^{-4}\ \mathrm{rad\,m^{-1}}$ indicating that this
wavelength will have the fastest growing instability.  The shape of the
eigenvectors (\fref{fig:EigenVectors}) can also be determined if that is of
interest.        

\begin{figure}[hbtp]
  \begin{center}
    \includegraphics[width=4in]{../python/VelocityProfile.pdf}
  \end{center}
    \caption{Background velocity profile for stability analysis.}
    \label{fig:VelocityProfile}
\end{figure}

\begin{figure}[hbtp]
  \begin{center}
    \includegraphics[width=4.5in]{../python/GrowthRate}
  \end{center}
  \caption{a) Fastest-growing eigenvalue and b) growth timescale as a function
of wavenumber for the velocity in \fref{fig:VelocityProfile}.  The fastest
growing wavenumber is broadly around 25-km scale.}
  \label{fig:GrowthRate}
\end{figure}

\begin{figure}[hbtp]
  \begin{center}
    \includegraphics[width=4in]{../python/EigenVectors.pdf}
  \end{center}
  \caption{Eigenvectors for discrete wavenumbers $k$.  The eigenvectors with
solid colours are unstable, and the dashed ones are stable.}
  \label{fig:EigenVectors}
\end{figure}

\clearpage
\subsection{Fully non-linear simulation}

The same velocity profile can be used as the initial conditions of a primitive
equation numerical model.  Here we set up a channel that has solid boundaries
in $y$ and is periodic in $x$ over a distance of 400 km (\fref{fig:ChanPar01})
and use the same velocity profile as \fref{fig:VelocityProfile}.  The flow is
seeded with very small imperfections that then grow or are suppressed.  We can
see the growth starting in \fref{fig:ChanPar01}b.  

\begin{figure}[hbtp]
  \begin{center}
\includegraphics[width=5in]{images/ChanPar040000324000}
  \end{center}
  \caption{Channel simulation with shear as in \fref{fig:VelocityProfile}, and
some added numerical noise after it has been allowed to evolve for 90 h.  Upper
plot is velocity, and lower plot is a passive tracer with the same initial
distribution as the velocity }
  \label{fig:ChanPar01}
\end{figure}

After the flow has evolved for longer (160h) than the fastest growth scale, it
is clear that a scale of about 25 km has been "selected" in the flow
instability (\fref{fig:ChanPar02}).  The other scales are still present, but
did not grow as fast as this scale and hence it dominates the figure.  Its
worth noting that the billows in this simulation are all slightly different,
and that is the chaotic amplification of slightly different-phased
disturbances.  

\begin{figure}[hbtp]
  \begin{center}
\includegraphics[width=5in]{images/ChanPar040000576000}
  \end{center}
  \caption{As \fref{fig:ChanPar01} except after 160 h, when the instabilites
have grown to noticeable size.}
  \label{fig:ChanPar02}
\end{figure}

\clearpage

As the flow evolves for a longer period of time, these un-even vortices start
to self-interact, and pairs of vortices merge to create larger channel-spanning
vortices (\fref{fig:ChanPar03}).  These eventually completely collapse into
near-homogeneous turbulence.

\begin{figure}[hbtp]
  \begin{center}
    \includegraphics[width=5in]{images/ChanPar040001008000}
  \end{center}
  \caption{As \fref{fig:ChanPar01} except after 160 h, when the instabilites
have grown to noticeable size.}
  \label{fig:ChanPar03}
\end{figure}
   
\clearpage

The evolving character of the flow is well-visualized by looking at spectra of
the tracer variance in the along-stream direction (\fref{fig:ShearSpectra}). 
Note that the fastest-growing wavenumbers are near the wavenumber predicted by
the linear perturbation theory ($k = 0.036\ \mathrm{cpkm} = 2.3\times10^{-4}\
\mathrm{rad/m}$).

\begin{figure}[hbtp]
  \begin{center}
    \includegraphics[width=5in]{images/ShearSpectra.pdf}
  \end{center}
  \caption{Spectra of tracer variance in the along-flow direction. Spectra with
more variance are from later in the simulation.  Note the broad peak near the
wavenumber of the fastest-growing mode (grey dashed line)}
  \label{fig:ShearSpectra}
\end{figure}

Note that there is a steep roll-off of the tracer spectrum to high wavenumbers.
This is a rough approximation of the inertial and convective subranges of
turbulence, where I say they are rough because the numerics are such that the
Reynolds number of these flows is not very high. 

\clearpage

\subsection{Stability of flows}

We chose the flow in the examples above to be unstable.  Consider a different
example (\fref{fig:RayleighInstabilityStable}) and note that the growth rate
for the unstable mode is many thousands of hours.  A primitive equation
simulation of this flow never develops instability because any disturbances
that form are killed by viscosity in the model before they have a chance to
grow.  Indeed the fact that we found a finite timescale for the growth of this
mode is just a numerical artifact, and can be made arbitrarily long if we
increase the vertical resolution of the numerical discretization.  

\begin{figure}[hbtp]
  \begin{center}
    \includegraphics[width=4in]{images/RayleighInstabilityStable.pdf}
  \end{center}
  \caption{Unstable mode growth rate for profile in the inset.  Note that it is
many thousands of hours, and hence effectively infinite.}
  \label{fig:RayleighInstabilityStable}
\end{figure}

It turns out that we can mathematically see which background flows $U(y)$ are
stable or unstable, which is of course very useful, and often one of the
primary goals of stability analysis.  Reconsider \fref{eq:Rayleigh} by
multiplying by the complex conjugate of $\hat{v}$, $v^*$ and integrate across
the y-domain:

\begin{equation}
    \int_{0}^L v^{*}\hat{v}'' \ \mathrm{d}y +  \int_{0}^L
v^{*}\hat{v}\left(\frac{U''}{c-U} - k^2 \right) \ \mathrm{d}y  = 0
\end{equation} 
Note that integrating by parts yields: $\int_{0}^L v^{*}\hat{v}''  =
v^{*}v'|^{L}_0 - \int_{0}^L |v'|^2\ \mathrm{d}z$ so that we can rewrite as 
\begin{equation}
    \int \left(|v_y|^2  + k^2|v|^2 \right)\ \mathrm{d}y + \int \frac{U''}{U-c}
|v|^2 \ \mathrm{d}y = 0
\end{equation}
The first term is always real.  The second term is only imaginary due to the
presence of $c$ in the denominator, so multiplying top and bottom by the
complex conjugate $U-c^*$, the imaginary part is
\begin{equation}
    c_{I}\int \frac{U'' |v|^2}{|U-c|^2}\ \mathrm{d}z = 0
\end{equation}
So, if $c_I$ is not zero then the integral must be zero, but the only thing
that can change sign is $U''$ so we can reach the conclusion that the flow is
stable is $U''$ does not change sign, i.e. if the curvature of $U(y)$ doesn't
change somewhere in the domain.  


Hence in the examples above, The first velocity profile
(\fref{fig:VelocityProfile}) \emph{can} be unstable because $U$ has negative
curvature above $y=0$ and positive below $y=0$.  Note that just because it
\emph{can} be unstable doesn't mean it will be, but the instability analysis
proves that it is.  Conversely, the velocity profile used in
\fref{fig:RayleighInstabilityStable} only has negative curvature, i.e. the sign
of $U''$ does not change in the channel.  Therefore that flow cannot be
unstable, and the numerical stability analysis and simulations reflect this. 



\section{Stratified Shear (Kelvin-Helmholtz) Instability}

\begin{figure}
\begin{center}
  \includegraphics[width=4.5in]{images/KHBillows}  
\end{center}
\caption{Kelvin-Helmholtz billows observed in the atmosphere, and visualized
with clouds}
\end{figure}

A second type of shear instability is the shear between two immiscible layers
$\rho_1$ and $\rho_2$ with a background flow that is at a different speed.  The
text follows a version where the layers are of infinite thickness, which is a
bit more analytically tractable than allowing variable depth like we did in
class.  For this case, we assume the flow on either side of the interface is
irrotational and divide into a base state and a fluctuating component.  Because
the flow is irrotational, the velocity potential in each layer follows
Laplace's equation.  The method of solution follows in the text
("Kelvin-Helmholtz Instability") and we won't repeat the math here.  However,
the same procedure is used as before, and the velocity potentials $phi_1$ and
$phi_2$ each have a vertical dependence that given the boundary conditions take
the form of decaying exponentials, and sinusoids in x and t.  One gets a
dispersion relation relating $c$ and $k$ for any disturbance:

\begin{equation}
  c  = \frac{\rho_2 U_2 + \rho_1 U_1}{\rho_2 + \rho_1} \pm
\left[\frac{g}{k}\frac{\rho_2 - \rho_1}{\rho_2+\rho_1} -
\rho_1\rho_2\left(\frac{U_1 - U_2}{\rho_1 + \rho_2} \right)^2 \right]^{1/2}
\end{equation}

which has imaginary roots, and thus is unstable if 
\begin{equation}
  g\left(\rho_2^2 - \rho_1^2 \right) < k\rho_1\rho_2\left(U_1-U_2\right)^2
\end{equation}

Note that for any given flow there is always a $k$ large enough that this is
true, but in practical cases, if the scale is small enough (high enough $k$)
then viscosity acts to suppress the growth of the instability.  Note the
exception is if $U_1=U_2$, in which case the flow is always stable if $\rho_2 >
\rho_1$.  

Also note that the stronger the density difference, the smaller scale the
growing mode.  Conversely, the stronger the shear $U_1-U_2$, the easier it is. 
So we often think of stratified shear instability as a competition between
shear producing the turbulence, and the stratification difference between the
layers suppressing it.  

Finally note that if $\rho_1 = \rho_2$, then flow is always unstable if the two
layers have different flow speeds ($c = (1/2)\left((U_1 + U_2)\pm
i(U_1-U_2)\right)$); such flows are called a "vortex sheet".  

As shown in the text, or videos posted on the course website, these
instabilitites are relatively easy to make both in nature and the lab.  Density
differences in the fluid naturally give rise to shear in the layers, whereas in
the homogenous shear instability discussed previously, the shear would have to
be imposed somehow, often via a splitter plate in the flow.  Nature doesn't
have many splitter plates, so stratified shear instability is a more natural
way to create instability.  

\section{Instability in viscous boundary layers}

We saw in the section on boundary layers that viscosity produces the Blasius
profile.  If such a profile were not subject to viscous effects, then the
profile that is formed would be stable.  However, viscosity is clearly
non-negligible in such flows.  Similarly, we usually think of viscosity as
suppressing the growth of instabilities because it damps the growth of the
perturbations that arise.

However, clearly at high enough Reynolds numbers this intuition breaks down and
boundary layers develop turbulence.  For pipe flow its found that for $Re>3000$
the flow will transition to turbulence, though if you are very careful the
transition can be delayed until the Reynolds number is another order of
magnitude larger.  The reason is almost certainly that the non-linear terms
matter and a linear perturbation analysis drops too much physics from the
problem.   The linear perturbation analysis is quite successful at inviscid
problems, but has trouble making simple predictions for viscous problems.




\end{document}

