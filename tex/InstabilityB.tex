\documentclass[11pt]{article}

\usepackage{graphicx}
\usepackage{fancyref}
\usepackage{fancyhdr}

\title{Physics 426 - Instability notes: Part B}
\author{Jody Klymak}

\begin{document}

\maketitle
\pagestyle{fancy}

\section{Stratified Shear (Kelvin-Helmholtz) Instability}

A second type of shear instability is the shear between two immiscible layers $\rho_1$ and $\rho_2$ with a background flow that is at a different speed.  The text follows a version where the layers are of infinite thickness, which is a bit more analytically tractable than allowing variable depth like we did in class.  For this case, we assume the flow on either side of the interface is irrotational and divide into a base state and a fluctuating component.  Because the flow is irrotational, the velocity potential in each layer follows Laplace's equation.  The method of solution follows in the text ("Kelvin-Helmholtz Instability") and we won't repeat the math here.  However, the same procedure is used as before, and the velocity potentials $phi_1$ and $phi_2$ each have a vertical dependence that given the boundary conditions take the form of decaying exponentials, and sinusoids in x and t.  One gets a dispersion relation relating $c$ and $k$ for any disturbance:

\begin{equation}
  c  = \frac{\rho_2 U_2 + \rho_1 U_1}{\rho_2 + \rho_1} \pm \left[\frac{g}{k}\frac{\rho_2 - \rho_1}{\rho_2+\rho_1} - \rho_1\rho_2\left(\frac{U_1 - U_2}{\rho_1 + \rho_2} \right)^2 \right]^{1/2}
\end{equation}

which has imaginary roots, and thus is unstable if 
\begin{equation}
  g\left(\rho_2^2 - \rho_1^2 \right) < k\rho_1\rho_2\left(U_1-U_2\right)^2
\end{equation}

Note that for any given flow there is always a $k$ large enough that this is true, but in practical cases, if the scale is small enough (high enough $k$) then viscosity acts to suppress the growth of the instability.  Note the exception is if $U_1=U_2$, in which case the flow is always stable if $\rho_2 > \rho_1$.  

Also note that the stronger the density difference, the smaller scale the growing mode.  Conversely, the stronger the shear $U_1-U_2$, the easier it is.  So we often think of stratified shear instability as a competition between shear producing the turbulence, and the stratification difference between the layers suppressing it.  

Finally note that if $\rho_1 = \rho_2$, then flow is always unstable if the two layers have different flow speeds ($c = (1/2)\left((U_1 + U_2)\pm i(U_1-U_2)\right)$); such flows are called a "vortex sheet".  

As shown in the text, or videos posted on the course website, these instabilitites are relatively easy to make both in nature and the lab.  Density differences in the fluid naturally give rise to shear in the layers, whereas in the homogenous shear instability discussed previously, the shear would have to be imposed somehow, often via a splitter plate in the flow.  Nature doesn't have many splitter plates, so stratified shear instability is a more natural way to create instability.  

\section{Instability in viscous boundary layers}

We saw in the section on boundary layers that viscosity produces the Blasius profile.  If such a profile were not subject to viscous effects, then the profile that is formed would be stable.  However, viscosity is clearly non-negligible in such flows.  Similarly, we usually think of viscosity as suppressing the growth of instabilities because it damps the growth of the perturbations that arise.

However, clearly at high enough Reynolds numbers this intuition breaks down and boundary layers develop turbulence.  For pipe flow its found that for $Re>3000$ the flow will transition to turbulence, though if you are very careful the transition can be delayed until the Reynolds number is another order of magnitude larger.  The reason is almost certainly that the non-linear terms matter and a linear perturbation analysis drops too much physics from the problem.   The linear perturbation analysis is quite successful at inviscid problems, but has trouble making simple predictions for viscous problems. 




\end{document}

